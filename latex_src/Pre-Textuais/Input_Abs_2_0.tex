%% =============================
%%      IMPORTANTE
%% ESTE ARQUIVO DEVE ESTAR SALVO COMO
%%      UTF - 8
%% =============================

% ----------------------------------------------------------
% Este capítulo é parte integrante do arquivo mestre
% Relatorio_TCC_Mestrado_Base_VERSÃO_SUBVERSÃO_FHZ
% ----------------------------------------------------------

% --------------------------------------------------------
% RESUMO - ABSTRACT - em arquivo separado - inserido com "\input{file}"
% --------------------------------------------------------

Métodos de aprendizado de máquina tornaram-se uma ferramenta poderosa para a comunidade acadêmica nas últimas décadas. No campo da fluidodinâmica computacional e da termodinâmica, esses métodos têm sido usados para realizar inferências rápidas sobre parâmetros não vistos, reduzindo a carga computacional associada aos métodos numéricos tradicionais. A maioria dos trabalhos nesse campo concentra-se em prever parâmetros globais ou integrados escalares. No entanto, este trabalho introduz um novo método de reconstrução de fluxo baseado em dados de aprendizado de máquina que pode reconstruir campos de fluxo inteiros usando a decomposição de valores singulares como técnica de redução de dimensionalidade. Além disso, conduzimos um estudo sobre o impacto do número de modos selecionados para redução de dimensionalidade nas métricas de desempenho geral e em outros hiperparâmetros para o desempenho de redes neurais. Também comparamos o desempenho de redes neurais com o método de Krigagem. Os principais resultados mostraram que redes neurais rasas com funções de ativação sigmoide tiveram um desempenho melhor do que redes neurais profundas, e o método de Krigagem foi mais rápido e preciso do que as redes neurais. Os melhores modelos obtidos até agora demonstraram sua viabilidade como modelos substitutos precisos.

%------------

% ----------------------------------------------------------
% Fim Arquivo