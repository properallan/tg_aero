%% =============================
%%      IMPORTANTE
%% ESTE ARQUIVO DEVE ESTAR SALVO COMO
%%      UTF - 8
%% =============================

% ----------------------------------------------------------
% Este capítulo é parte integrante do arquivo mestre
% Relatorio_TCC_Mestrado_Base_VERSÃO_SUBVERSÃO
% Formatos com \caption acima do \includegraphics conforme ABNT
% ----------------------------------------------------------

% -----------------------------------------
% Metodologia
%
% A ideia é utilizar a mesma sequência de entrada ter a opção de trocar a posição conforme a necessidade
% 
% O comando deve ser usado dentro do ambiente {figure} de forma que o comando \label{key} esteja dentro do ambiente e permita a melhor localização pelo editor LaTeX
%
% Forma de uso
%
% --------------------- Figuras Simples
%	\begin{figure}[H]
%		\figNameCommand{options includegraphics}{filename with path}{caption}
%		\label{label_figure_1}
%	\end{figure}
%
% --------------------- Figuras Compostas
%	\begin{figure}[H]
%		\figCapTwoSubfigInc{caption geral}
%		{subcaption (a) }
%		{width = 7cm, height = 6cm, trim = 0cm 0cm 0cm 0cm, clip = true}
%		{figure with path}
%		{subcaption (b) }
%		{width = 7cm, height = 6cm, trim = 0cm 0cm 0cm 0cm, clip = true}
%		{figure with path}
%		\label{fig_label}
%	\end{figure}
%
% --------------------- Comando Trim
% trim = left bottom right top
% -----------------------------------------

% -----------------------------------------
% Comandos para Figuras Simples
% -----------------------------------------
\newcommand{\figIncLongCap}[3]
{ 	
	\centering
	\caption{#3}
	\includegraphics[#1]{#2}
}

\newcommand{\figIncShortCap}[4]
{ 	
	\centering
	\caption[#4]{#3}
	\includegraphics[#1]{#2}
}

% -----------------------------------------
% Comandos para Figuras Duplas Com Subfigures
% -----------------------------------------
\newcommand{\figTwoSubfigIncLongCap}[7]
{ 	
	\centering
	\caption{#7}
	\subfigure[#1]
	{\includegraphics[#2]{#3}}
	\quad
	\subfigure[#4]
	{\includegraphics[#5]{#6}}
}

\newcommand{\figTwoSubfigIncShortCap}[8]
{ 	
	\centering
	\caption[#8]{#7}
	\subfigure[#1]
	{\includegraphics[#2]{#3}}
	\quad
	\subfigure[#4]
	{\includegraphics[#5]{#6}}
}

% -----------------------------------------
% Comandos para Figuras Triplas em Cascata Com Subfigures
% -----------------------------------------
\newcommand{\figThreeSubfigIncCap}[4]
{ 	
	\centering
	\SubfigSubcapSubfigInc{#1}
	\quad
	\SubfigSubcapSubfigInc{#2}
	\quad
	\SubfigSubcapSubfigInc{#3}
	\quad
	\caption{#4}
}

% -----------------------------------------
% Comandos Base para Figuras Múltiplas em Cascata Com Subfigures
% -----------------------------------------
\newcommand{\SubfigcapSubfigInc}[3]
{ 	
	\subfigure[#1]
	{\includegraphics[#2]{#3}} 
}

%%%%%%%%%%%%%%%%%%%%%%%%%%%%%%%%%%
% ===== Magno old versions ===== %
%%%%%%%%%%%%%%%%%%%%%%%%%%%%%%%%%%

% -----------------------------------------
% Comandos para Figuras Duplas sem Subfigure
% -----------------------------------------
\newcommand{\figDob}[5]
{   
	\centering
	\caption{#5}
	\includegraphics[#1]{#2}
	\includegraphics[#3]{#4} \\
	\vspace*{0.2cm}
	(a) \hspace{6.5cm} (b)
}

% -----------------------------------------
% Comandos para Figuras Triplas sem Subfigure
% -----------------------------------------
\newcommand{\figTre}[7]
{   
	\centering
	\caption{#7}
	\includegraphics[#1]{#2}
	\includegraphics[#3]{#4} \\
	\vspace*{0.2cm}
	(a) \hspace{6.5cm} (b) \\
	\includegraphics[#5]{#6} \\
	\vspace*{0.2cm}
	(c)
}

\newcommand{\figTreV}[7]
{   
	\centering
	\caption{#7}
	\includegraphics[#1]{#2}
	\includegraphics[#3]{#4}
	\includegraphics[#5]{#6}   \\
	\vspace*{0.2cm}
	(a) \hspace{4.cm} (b) \hspace{4.cm} (c)
}
% ----------------------------------------------------------
% Fim Arquivo