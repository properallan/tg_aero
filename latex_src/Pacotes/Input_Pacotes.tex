%% =============================
%%      IMPORTANTE
%% ESTE ARQUIVO DEVE ESTAR SALVO COMO
%%      UTF - 8
%% =============================

% ----------------------------------------------------------
% Este capítulo é parte integrante do arquivo mestre
% Relatorio_TCC_Mestrado_Base_VERSÃO_SUBVERSÃO
% ----------------------------------------------------------
\usepackage{natbib}
\usepackage[T1]{fontenc}
%\usepackage[latin1]{inputenc}
\usepackage[utf8]{inputenc}
% =========
\usepackage{pgfplots}
\usepackage{tabularx}
% -----------------------------------------
% Pacotes básicos 
% -----------------------------------------
\usepackage[section]{placeins}
\usepackage{lastpage}		% Usado pela Ficha catalográfica
\usepackage{indentfirst}	% Indenta o primeiro parágrafo de cada seção.
\usepackage{color}			% Controle das cores
\usepackage{microtype} 		% Melhorias de justificação \usepackage[section]{placeins}
% \usepackage{amsmath} 		% Referências de equações com parênteses automático usar \eqref
\usepackage{physics,amsmath}

\usepackage{amssymb} 		% Símbolos matemáticos, incluindo os principais conjuntos numéricos
\usepackage{cancel}
% -----------------------------------------

% -----------------------------------------
% Pacotes gráficos
% -----------------------------------------
\usepackage{ae}
\usepackage{aecompl}
\usepackage{graphicx, import}	% Inclusão de gráficos
\usepackage{float} 		% Permite que eu use "H" para a figura ficar entre os parágrafos que eu quero
%\usepackage{subfigure} % make it possible to include more than one captioned figure/table in a single float
%\usepackage[font=small,labelfont=bf]{caption}
\usepackage[font=footnotesize]{caption} % Reduz o tamanho de todos os ``captions''
%\usepackage{subcaption}
%\usepackage{subfloat} % make it possible to include more than one captioned 
\usepackage{epstopdf}
\usepackage{ulem}
\usepackage{nomencl}
\makenomenclature
% \usepackage{framed}
\usepackage{lipsum} % geração de texto inútil - dummy
\usepackage{morefloats}

% -----------------------------------------
% Pacotes gráficos - Formatação do título
% -----------------------------------------
% Se fncychap for adicionado como RequirePackage dentro de ufabcFHZ#.cls não gera o mesmo efeito.
\usepackage[Bjornstrup]{fncychap} % Formas elegantes de cabeçalho
%% ------------------------------------
% Referência: http://tex.stackexchange.com/questions/13357/fncychap-package-reduce-vertical-gap-space-between-header-and-chapter-heading
%% ------------------------------------
%% A única alteração feita é em ``\vspace'', por padrão será usado ``-1cm'', fica bom, maximiza o espaço  - Não é necessário editar esse trecho.
%% ------------------------------------
\makeatletter
\renewcommand*{\@makechapterhead}[1]{%
	%\vspace*{-5\p@}%
	\vspace*{-1cm}% Recuo vertical para maximizar aproveitamento da página
	{\parindent \z@ \raggedright \normalfont
		\ifnum \c@secnumdepth >\m@ne
		\if@mainmatter%%%%% Fix for frontmatter, mainmatter, and backmatter 040920
		\DOCH
		\fi
		\fi
		\interlinepenalty\@M
		\if@mainmatter%%%%% Fix for frontmatter, mainmatter, and backmatter 060424
		\DOTI{#1}%
		\else%
		\DOTIS{#1}%
		\fi
	}
}
% For the case \chapter*:
\renewcommand*{\@makeschapterhead}[1]{%
	\vspace*{-1cm}% Recuo vertical para maximizar aproveitamento da página
	{\parindent \z@ \raggedright
		\normalfont
		\interlinepenalty\@M
		\DOTIS{#1}
		\vskip 40\p@
	}
}
\makeatother
%% ------------------------------------

% -----------------------------------------
% Pacotes Ambiente de comentário
% -----------------------------------------
\usepackage{comment}

% ----------------------------------------- 
% Tabelas
% ----------------------------------------- 
\usepackage{booktabs} % for much better looking tables
\usepackage{multirow} % Permite uma célula de várias linhas
% ----------------------------------------- 

% ----------------------------------------- 
% Enumerate com letras pré-fixas
% ----------------------------------------- 
\usepackage{paralist} 		% very flexible & customisable lists (eg. enumerate/itemize, etc.)
\usepackage{enumitem}

% ----------------------------------------- 
% Equações
% ----------------------------------------- 
\usepackage{breqn} 		 % Garante quebra automático com \begin{dmath}, mesmo contador de \begin{equation}
\usepackage{array} 		 % for better arrays (eg matrices) in maths
\usepackage{subeqnarray} % Permite o use de subnumeração numa equação 1.a 1.b 1.c etc
\usepackage{cancel} 	 % Permite o corte numa simplificação de expressão:
% \cancel{expression}
% \cancelto{value}{expression}

% ----------------------------------------- 
% Links Coloridos - Com seleção de uso de cores pelo usuário no arquivo base usando \toogletrue ou \togglefalse
% ----------------------------------------- 
\usepackage[colorlinks=true,
linkcolor 		= red,
anchorcolor 	= black,
citecolor 		= green,
filecolor 		= cyan,
	% ==== Selecionando opção de links coloridos - Em parceria com comando "\newtoggle{LinksComCores}"
	\iftoggle{LinksComCores}{%
		%hidelinks
	}{%
		hidelinks
	}
	% ==== 
]{hyperref} %Pacote para hyperlinks
%hidelinks %opção para os links não serem coloridos, útil para a versão final do trabalho, também posso usar %linkcolor=black


% ----------------------------------------- 
% Links Coloridos - Com seleção de uso de cores (des)comentando "hidelinks"
%% - Este código está dentro da classe ufabcFHZ#.cls
%% - 	Ele é mantido aqui para conferência
% ----------------------------------------- 
%\usepackage[colorlinks=true,
%linkcolor 		= red,
%anchorcolor 	= black,
%citecolor 		= green,
%filecolor 		= cyan,
%hidelinks
%]{hyperref} %Pacote para hyperlinks
%%hidelinks %opção para os links não serem coloridos, útil para a versão final do trabalho, também posso usar %linkcolor=black

% ----------------------------------------- 
% Inserção de código do Matlab
% ----------------------------------------- 
\usepackage[numbered]{mcode} 		% configure listings for Matlab

%==== Pacote para url nas referências da ABNT
%\usepackage[num,abnt-url-package=url]{abntcite}
%====

% for subfigures
\usepackage{caption}
\usepackage{subcaption}
\usepackage{capt-of}

% to draw circuits
\usepackage{circuitikz}
\usepackage{tikz}

\usepackage{makecell}
% ----------------------------------------------------------
% Fim Arquivo

\usetikzlibrary{shapes.geometric,arrows.meta}
\usetikzlibrary{positioning}
\usetikzlibrary{calc}
\usetikzlibrary{backgrounds}
\usetikzlibrary{fit}

\tikzstyle{diam} = [diamond, aspect=2, draw, fill=red!40, text width=6em,text centered ]
\tikzstyle{block} = [rectangle, draw, fill=blue!20, text width=3cm,text centered, rounded corners, minimum height=2em ]
\tikzstyle{trap} = [trapezium, trapezium left angle=70, trapezium right angle=110, minimum height=2em, text centered, draw=red, fill=green!30]
\tikzstyle{rect} = [rectangle, minimum width=3cm, minimum height=1cm, text centered, draw=red, fill=orange!30]
\tikzstyle{line} = [draw, -latex]


% for pseudo codes
\usepackage{algorithm}
\usepackage[noend]{algpseudocode}

\makeatletter
\newcommand*{\algrule}[1][\algorithmicindent]{%
  \makebox[#1][l]{%
    \hspace*{.2em}% <------------- This is where the rule starts from
    \vrule height .75\baselineskip depth .25\baselineskip
  }
}
\newcount\ALG@printindent@tempcnta
\def\ALG@printindent{%
    \ifnum \theALG@nested>0% is there anything to print
    \ifx\ALG@text\ALG@x@notext% is this an end group without any text?
    % do nothing
    \else
    \unskip
    % draw a rule for each indent level
    \ALG@printindent@tempcnta=1
    \loop
    \algrule[\csname ALG@ind@\the\ALG@printindent@tempcnta\endcsname]%
    \advance \ALG@printindent@tempcnta 1
    \ifnum \ALG@printindent@tempcnta<\numexpr\theALG@nested+1\relax
    \repeat
    \fi
    \fi
}
% the following line injects our new indent handling code in place of the default spacing
\patchcmd{\ALG@doentity}{\noindent\hskip\ALG@tlm}{\ALG@printindent}{}{\errmessage{failed to patch}}
\patchcmd{\ALG@doentity}{\item[]\nointerlineskip}{}{}{} % no spurious vertical space
% end vertical rule patch for algorithmicx
\makeatother

% mathcal lowerletters
%\usepackage{dutchcal}
